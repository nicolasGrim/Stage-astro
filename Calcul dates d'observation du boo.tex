\documentclass[11pt]{article}

    \usepackage[breakable]{tcolorbox}
    \usepackage{parskip} % Stop auto-indenting (to mimic markdown behaviour)
    
    \usepackage{iftex}
    \ifPDFTeX
    	\usepackage[T1]{fontenc}
    	\usepackage[utf8]{inputenc}
    	\usepackage{mathpazo}
    \else
    	\usepackage{fontspec}
    \fi

    % Basic figure setup, for now with no caption control since it's done
    % automatically by Pandoc (which extracts ![](path) syntax from Markdown).
    \usepackage{graphicx}
    % Maintain compatibility with old templates. Remove in nbconvert 6.0
    \let\Oldincludegraphics\includegraphics
    % Ensure that by default, figures have no caption (until we provide a
    % proper Figure object with a Caption API and a way to capture that
    % in the conversion process - todo).
    \usepackage{caption}
    \DeclareCaptionFormat{nocaption}{}
    \captionsetup{format=nocaption,aboveskip=0pt,belowskip=0pt}

    \usepackage[Export]{adjustbox} % Used to constrain images to a maximum size
    \adjustboxset{max size={0.9\linewidth}{0.9\paperheight}}
    \usepackage{float}
    \floatplacement{figure}{H} % forces figures to be placed at the correct location
    \usepackage{xcolor} % Allow colors to be defined
    \usepackage{enumerate} % Needed for markdown enumerations to work
    \usepackage{geometry} % Used to adjust the document margins
    \usepackage{amsmath} % Equations
    \usepackage{amssymb} % Equations
    \usepackage{textcomp} % defines textquotesingle
    % Hack from http://tex.stackexchange.com/a/47451/13684:
    \AtBeginDocument{%
        \def\PYZsq{\textquotesingle}% Upright quotes in Pygmentized code
    }
    \usepackage{upquote} % Upright quotes for verbatim code
    \usepackage{eurosym} % defines \euro
    \usepackage[mathletters]{ucs} % Extended unicode (utf-8) support
    \usepackage{fancyvrb} % verbatim replacement that allows latex
    \usepackage{grffile} % extends the file name processing of package graphics 
                         % to support a larger range
    \makeatletter % fix for grffile with XeLaTeX
    \def\Gread@@xetex#1{%
      \IfFileExists{"\Gin@base".bb}%
      {\Gread@eps{\Gin@base.bb}}%
      {\Gread@@xetex@aux#1}%
    }
    \makeatother

    % The hyperref package gives us a pdf with properly built
    % internal navigation ('pdf bookmarks' for the table of contents,
    % internal cross-reference links, web links for URLs, etc.)
    \usepackage{hyperref}
    % The default LaTeX title has an obnoxious amount of whitespace. By default,
    % titling removes some of it. It also provides customization options.
    \usepackage{titling}
    \usepackage{longtable} % longtable support required by pandoc >1.10
    \usepackage{booktabs}  % table support for pandoc > 1.12.2
    \usepackage[inline]{enumitem} % IRkernel/repr support (it uses the enumerate* environment)
    \usepackage[normalem]{ulem} % ulem is needed to support strikethroughs (\sout)
                                % normalem makes italics be italics, not underlines
    \usepackage{mathrsfs}
    

    
    % Colors for the hyperref package
    \definecolor{urlcolor}{rgb}{0,.145,.698}
    \definecolor{linkcolor}{rgb}{.71,0.21,0.01}
    \definecolor{citecolor}{rgb}{.12,.54,.11}

    % ANSI colors
    \definecolor{ansi-black}{HTML}{3E424D}
    \definecolor{ansi-black-intense}{HTML}{282C36}
    \definecolor{ansi-red}{HTML}{E75C58}
    \definecolor{ansi-red-intense}{HTML}{B22B31}
    \definecolor{ansi-green}{HTML}{00A250}
    \definecolor{ansi-green-intense}{HTML}{007427}
    \definecolor{ansi-yellow}{HTML}{DDB62B}
    \definecolor{ansi-yellow-intense}{HTML}{B27D12}
    \definecolor{ansi-blue}{HTML}{208FFB}
    \definecolor{ansi-blue-intense}{HTML}{0065CA}
    \definecolor{ansi-magenta}{HTML}{D160C4}
    \definecolor{ansi-magenta-intense}{HTML}{A03196}
    \definecolor{ansi-cyan}{HTML}{60C6C8}
    \definecolor{ansi-cyan-intense}{HTML}{258F8F}
    \definecolor{ansi-white}{HTML}{C5C1B4}
    \definecolor{ansi-white-intense}{HTML}{A1A6B2}
    \definecolor{ansi-default-inverse-fg}{HTML}{FFFFFF}
    \definecolor{ansi-default-inverse-bg}{HTML}{000000}

    % commands and environments needed by pandoc snippets
    % extracted from the output of `pandoc -s`
    \providecommand{\tightlist}{%
      \setlength{\itemsep}{0pt}\setlength{\parskip}{0pt}}
    \DefineVerbatimEnvironment{Highlighting}{Verbatim}{commandchars=\\\{\}}
    % Add ',fontsize=\small' for more characters per line
    \newenvironment{Shaded}{}{}
    \newcommand{\KeywordTok}[1]{\textcolor[rgb]{0.00,0.44,0.13}{\textbf{{#1}}}}
    \newcommand{\DataTypeTok}[1]{\textcolor[rgb]{0.56,0.13,0.00}{{#1}}}
    \newcommand{\DecValTok}[1]{\textcolor[rgb]{0.25,0.63,0.44}{{#1}}}
    \newcommand{\BaseNTok}[1]{\textcolor[rgb]{0.25,0.63,0.44}{{#1}}}
    \newcommand{\FloatTok}[1]{\textcolor[rgb]{0.25,0.63,0.44}{{#1}}}
    \newcommand{\CharTok}[1]{\textcolor[rgb]{0.25,0.44,0.63}{{#1}}}
    \newcommand{\StringTok}[1]{\textcolor[rgb]{0.25,0.44,0.63}{{#1}}}
    \newcommand{\CommentTok}[1]{\textcolor[rgb]{0.38,0.63,0.69}{\textit{{#1}}}}
    \newcommand{\OtherTok}[1]{\textcolor[rgb]{0.00,0.44,0.13}{{#1}}}
    \newcommand{\AlertTok}[1]{\textcolor[rgb]{1.00,0.00,0.00}{\textbf{{#1}}}}
    \newcommand{\FunctionTok}[1]{\textcolor[rgb]{0.02,0.16,0.49}{{#1}}}
    \newcommand{\RegionMarkerTok}[1]{{#1}}
    \newcommand{\ErrorTok}[1]{\textcolor[rgb]{1.00,0.00,0.00}{\textbf{{#1}}}}
    \newcommand{\NormalTok}[1]{{#1}}
    
    % Additional commands for more recent versions of Pandoc
    \newcommand{\ConstantTok}[1]{\textcolor[rgb]{0.53,0.00,0.00}{{#1}}}
    \newcommand{\SpecialCharTok}[1]{\textcolor[rgb]{0.25,0.44,0.63}{{#1}}}
    \newcommand{\VerbatimStringTok}[1]{\textcolor[rgb]{0.25,0.44,0.63}{{#1}}}
    \newcommand{\SpecialStringTok}[1]{\textcolor[rgb]{0.73,0.40,0.53}{{#1}}}
    \newcommand{\ImportTok}[1]{{#1}}
    \newcommand{\DocumentationTok}[1]{\textcolor[rgb]{0.73,0.13,0.13}{\textit{{#1}}}}
    \newcommand{\AnnotationTok}[1]{\textcolor[rgb]{0.38,0.63,0.69}{\textbf{\textit{{#1}}}}}
    \newcommand{\CommentVarTok}[1]{\textcolor[rgb]{0.38,0.63,0.69}{\textbf{\textit{{#1}}}}}
    \newcommand{\VariableTok}[1]{\textcolor[rgb]{0.10,0.09,0.49}{{#1}}}
    \newcommand{\ControlFlowTok}[1]{\textcolor[rgb]{0.00,0.44,0.13}{\textbf{{#1}}}}
    \newcommand{\OperatorTok}[1]{\textcolor[rgb]{0.40,0.40,0.40}{{#1}}}
    \newcommand{\BuiltInTok}[1]{{#1}}
    \newcommand{\ExtensionTok}[1]{{#1}}
    \newcommand{\PreprocessorTok}[1]{\textcolor[rgb]{0.74,0.48,0.00}{{#1}}}
    \newcommand{\AttributeTok}[1]{\textcolor[rgb]{0.49,0.56,0.16}{{#1}}}
    \newcommand{\InformationTok}[1]{\textcolor[rgb]{0.38,0.63,0.69}{\textbf{\textit{{#1}}}}}
    \newcommand{\WarningTok}[1]{\textcolor[rgb]{0.38,0.63,0.69}{\textbf{\textit{{#1}}}}}
    
    
    % Define a nice break command that doesn't care if a line doesn't already
    % exist.
    \def\br{\hspace*{\fill} \\* }
    % Math Jax compatibility definitions
    \def\gt{>}
    \def\lt{<}
    \let\Oldtex\TeX
    \let\Oldlatex\LaTeX
    \renewcommand{\TeX}{\textrm{\Oldtex}}
    \renewcommand{\LaTeX}{\textrm{\Oldlatex}}
    % Document parameters
    % Document title
    \title{Calcul dates d'observation du boo}
    
    
    
    
    
% Pygments definitions
\makeatletter
\def\PY@reset{\let\PY@it=\relax \let\PY@bf=\relax%
    \let\PY@ul=\relax \let\PY@tc=\relax%
    \let\PY@bc=\relax \let\PY@ff=\relax}
\def\PY@tok#1{\csname PY@tok@#1\endcsname}
\def\PY@toks#1+{\ifx\relax#1\empty\else%
    \PY@tok{#1}\expandafter\PY@toks\fi}
\def\PY@do#1{\PY@bc{\PY@tc{\PY@ul{%
    \PY@it{\PY@bf{\PY@ff{#1}}}}}}}
\def\PY#1#2{\PY@reset\PY@toks#1+\relax+\PY@do{#2}}

\expandafter\def\csname PY@tok@w\endcsname{\def\PY@tc##1{\textcolor[rgb]{0.73,0.73,0.73}{##1}}}
\expandafter\def\csname PY@tok@c\endcsname{\let\PY@it=\textit\def\PY@tc##1{\textcolor[rgb]{0.25,0.50,0.50}{##1}}}
\expandafter\def\csname PY@tok@cp\endcsname{\def\PY@tc##1{\textcolor[rgb]{0.74,0.48,0.00}{##1}}}
\expandafter\def\csname PY@tok@k\endcsname{\let\PY@bf=\textbf\def\PY@tc##1{\textcolor[rgb]{0.00,0.50,0.00}{##1}}}
\expandafter\def\csname PY@tok@kp\endcsname{\def\PY@tc##1{\textcolor[rgb]{0.00,0.50,0.00}{##1}}}
\expandafter\def\csname PY@tok@kt\endcsname{\def\PY@tc##1{\textcolor[rgb]{0.69,0.00,0.25}{##1}}}
\expandafter\def\csname PY@tok@o\endcsname{\def\PY@tc##1{\textcolor[rgb]{0.40,0.40,0.40}{##1}}}
\expandafter\def\csname PY@tok@ow\endcsname{\let\PY@bf=\textbf\def\PY@tc##1{\textcolor[rgb]{0.67,0.13,1.00}{##1}}}
\expandafter\def\csname PY@tok@nb\endcsname{\def\PY@tc##1{\textcolor[rgb]{0.00,0.50,0.00}{##1}}}
\expandafter\def\csname PY@tok@nf\endcsname{\def\PY@tc##1{\textcolor[rgb]{0.00,0.00,1.00}{##1}}}
\expandafter\def\csname PY@tok@nc\endcsname{\let\PY@bf=\textbf\def\PY@tc##1{\textcolor[rgb]{0.00,0.00,1.00}{##1}}}
\expandafter\def\csname PY@tok@nn\endcsname{\let\PY@bf=\textbf\def\PY@tc##1{\textcolor[rgb]{0.00,0.00,1.00}{##1}}}
\expandafter\def\csname PY@tok@ne\endcsname{\let\PY@bf=\textbf\def\PY@tc##1{\textcolor[rgb]{0.82,0.25,0.23}{##1}}}
\expandafter\def\csname PY@tok@nv\endcsname{\def\PY@tc##1{\textcolor[rgb]{0.10,0.09,0.49}{##1}}}
\expandafter\def\csname PY@tok@no\endcsname{\def\PY@tc##1{\textcolor[rgb]{0.53,0.00,0.00}{##1}}}
\expandafter\def\csname PY@tok@nl\endcsname{\def\PY@tc##1{\textcolor[rgb]{0.63,0.63,0.00}{##1}}}
\expandafter\def\csname PY@tok@ni\endcsname{\let\PY@bf=\textbf\def\PY@tc##1{\textcolor[rgb]{0.60,0.60,0.60}{##1}}}
\expandafter\def\csname PY@tok@na\endcsname{\def\PY@tc##1{\textcolor[rgb]{0.49,0.56,0.16}{##1}}}
\expandafter\def\csname PY@tok@nt\endcsname{\let\PY@bf=\textbf\def\PY@tc##1{\textcolor[rgb]{0.00,0.50,0.00}{##1}}}
\expandafter\def\csname PY@tok@nd\endcsname{\def\PY@tc##1{\textcolor[rgb]{0.67,0.13,1.00}{##1}}}
\expandafter\def\csname PY@tok@s\endcsname{\def\PY@tc##1{\textcolor[rgb]{0.73,0.13,0.13}{##1}}}
\expandafter\def\csname PY@tok@sd\endcsname{\let\PY@it=\textit\def\PY@tc##1{\textcolor[rgb]{0.73,0.13,0.13}{##1}}}
\expandafter\def\csname PY@tok@si\endcsname{\let\PY@bf=\textbf\def\PY@tc##1{\textcolor[rgb]{0.73,0.40,0.53}{##1}}}
\expandafter\def\csname PY@tok@se\endcsname{\let\PY@bf=\textbf\def\PY@tc##1{\textcolor[rgb]{0.73,0.40,0.13}{##1}}}
\expandafter\def\csname PY@tok@sr\endcsname{\def\PY@tc##1{\textcolor[rgb]{0.73,0.40,0.53}{##1}}}
\expandafter\def\csname PY@tok@ss\endcsname{\def\PY@tc##1{\textcolor[rgb]{0.10,0.09,0.49}{##1}}}
\expandafter\def\csname PY@tok@sx\endcsname{\def\PY@tc##1{\textcolor[rgb]{0.00,0.50,0.00}{##1}}}
\expandafter\def\csname PY@tok@m\endcsname{\def\PY@tc##1{\textcolor[rgb]{0.40,0.40,0.40}{##1}}}
\expandafter\def\csname PY@tok@gh\endcsname{\let\PY@bf=\textbf\def\PY@tc##1{\textcolor[rgb]{0.00,0.00,0.50}{##1}}}
\expandafter\def\csname PY@tok@gu\endcsname{\let\PY@bf=\textbf\def\PY@tc##1{\textcolor[rgb]{0.50,0.00,0.50}{##1}}}
\expandafter\def\csname PY@tok@gd\endcsname{\def\PY@tc##1{\textcolor[rgb]{0.63,0.00,0.00}{##1}}}
\expandafter\def\csname PY@tok@gi\endcsname{\def\PY@tc##1{\textcolor[rgb]{0.00,0.63,0.00}{##1}}}
\expandafter\def\csname PY@tok@gr\endcsname{\def\PY@tc##1{\textcolor[rgb]{1.00,0.00,0.00}{##1}}}
\expandafter\def\csname PY@tok@ge\endcsname{\let\PY@it=\textit}
\expandafter\def\csname PY@tok@gs\endcsname{\let\PY@bf=\textbf}
\expandafter\def\csname PY@tok@gp\endcsname{\let\PY@bf=\textbf\def\PY@tc##1{\textcolor[rgb]{0.00,0.00,0.50}{##1}}}
\expandafter\def\csname PY@tok@go\endcsname{\def\PY@tc##1{\textcolor[rgb]{0.53,0.53,0.53}{##1}}}
\expandafter\def\csname PY@tok@gt\endcsname{\def\PY@tc##1{\textcolor[rgb]{0.00,0.27,0.87}{##1}}}
\expandafter\def\csname PY@tok@err\endcsname{\def\PY@bc##1{\setlength{\fboxsep}{0pt}\fcolorbox[rgb]{1.00,0.00,0.00}{1,1,1}{\strut ##1}}}
\expandafter\def\csname PY@tok@kc\endcsname{\let\PY@bf=\textbf\def\PY@tc##1{\textcolor[rgb]{0.00,0.50,0.00}{##1}}}
\expandafter\def\csname PY@tok@kd\endcsname{\let\PY@bf=\textbf\def\PY@tc##1{\textcolor[rgb]{0.00,0.50,0.00}{##1}}}
\expandafter\def\csname PY@tok@kn\endcsname{\let\PY@bf=\textbf\def\PY@tc##1{\textcolor[rgb]{0.00,0.50,0.00}{##1}}}
\expandafter\def\csname PY@tok@kr\endcsname{\let\PY@bf=\textbf\def\PY@tc##1{\textcolor[rgb]{0.00,0.50,0.00}{##1}}}
\expandafter\def\csname PY@tok@bp\endcsname{\def\PY@tc##1{\textcolor[rgb]{0.00,0.50,0.00}{##1}}}
\expandafter\def\csname PY@tok@fm\endcsname{\def\PY@tc##1{\textcolor[rgb]{0.00,0.00,1.00}{##1}}}
\expandafter\def\csname PY@tok@vc\endcsname{\def\PY@tc##1{\textcolor[rgb]{0.10,0.09,0.49}{##1}}}
\expandafter\def\csname PY@tok@vg\endcsname{\def\PY@tc##1{\textcolor[rgb]{0.10,0.09,0.49}{##1}}}
\expandafter\def\csname PY@tok@vi\endcsname{\def\PY@tc##1{\textcolor[rgb]{0.10,0.09,0.49}{##1}}}
\expandafter\def\csname PY@tok@vm\endcsname{\def\PY@tc##1{\textcolor[rgb]{0.10,0.09,0.49}{##1}}}
\expandafter\def\csname PY@tok@sa\endcsname{\def\PY@tc##1{\textcolor[rgb]{0.73,0.13,0.13}{##1}}}
\expandafter\def\csname PY@tok@sb\endcsname{\def\PY@tc##1{\textcolor[rgb]{0.73,0.13,0.13}{##1}}}
\expandafter\def\csname PY@tok@sc\endcsname{\def\PY@tc##1{\textcolor[rgb]{0.73,0.13,0.13}{##1}}}
\expandafter\def\csname PY@tok@dl\endcsname{\def\PY@tc##1{\textcolor[rgb]{0.73,0.13,0.13}{##1}}}
\expandafter\def\csname PY@tok@s2\endcsname{\def\PY@tc##1{\textcolor[rgb]{0.73,0.13,0.13}{##1}}}
\expandafter\def\csname PY@tok@sh\endcsname{\def\PY@tc##1{\textcolor[rgb]{0.73,0.13,0.13}{##1}}}
\expandafter\def\csname PY@tok@s1\endcsname{\def\PY@tc##1{\textcolor[rgb]{0.73,0.13,0.13}{##1}}}
\expandafter\def\csname PY@tok@mb\endcsname{\def\PY@tc##1{\textcolor[rgb]{0.40,0.40,0.40}{##1}}}
\expandafter\def\csname PY@tok@mf\endcsname{\def\PY@tc##1{\textcolor[rgb]{0.40,0.40,0.40}{##1}}}
\expandafter\def\csname PY@tok@mh\endcsname{\def\PY@tc##1{\textcolor[rgb]{0.40,0.40,0.40}{##1}}}
\expandafter\def\csname PY@tok@mi\endcsname{\def\PY@tc##1{\textcolor[rgb]{0.40,0.40,0.40}{##1}}}
\expandafter\def\csname PY@tok@il\endcsname{\def\PY@tc##1{\textcolor[rgb]{0.40,0.40,0.40}{##1}}}
\expandafter\def\csname PY@tok@mo\endcsname{\def\PY@tc##1{\textcolor[rgb]{0.40,0.40,0.40}{##1}}}
\expandafter\def\csname PY@tok@ch\endcsname{\let\PY@it=\textit\def\PY@tc##1{\textcolor[rgb]{0.25,0.50,0.50}{##1}}}
\expandafter\def\csname PY@tok@cm\endcsname{\let\PY@it=\textit\def\PY@tc##1{\textcolor[rgb]{0.25,0.50,0.50}{##1}}}
\expandafter\def\csname PY@tok@cpf\endcsname{\let\PY@it=\textit\def\PY@tc##1{\textcolor[rgb]{0.25,0.50,0.50}{##1}}}
\expandafter\def\csname PY@tok@c1\endcsname{\let\PY@it=\textit\def\PY@tc##1{\textcolor[rgb]{0.25,0.50,0.50}{##1}}}
\expandafter\def\csname PY@tok@cs\endcsname{\let\PY@it=\textit\def\PY@tc##1{\textcolor[rgb]{0.25,0.50,0.50}{##1}}}

\def\PYZbs{\char`\\}
\def\PYZus{\char`\_}
\def\PYZob{\char`\{}
\def\PYZcb{\char`\}}
\def\PYZca{\char`\^}
\def\PYZam{\char`\&}
\def\PYZlt{\char`\<}
\def\PYZgt{\char`\>}
\def\PYZsh{\char`\#}
\def\PYZpc{\char`\%}
\def\PYZdl{\char`\$}
\def\PYZhy{\char`\-}
\def\PYZsq{\char`\'}
\def\PYZdq{\char`\"}
\def\PYZti{\char`\~}
% for compatibility with earlier versions
\def\PYZat{@}
\def\PYZlb{[}
\def\PYZrb{]}
\makeatother


    % For linebreaks inside Verbatim environment from package fancyvrb. 
    \makeatletter
        \newbox\Wrappedcontinuationbox 
        \newbox\Wrappedvisiblespacebox 
        \newcommand*\Wrappedvisiblespace {\textcolor{red}{\textvisiblespace}} 
        \newcommand*\Wrappedcontinuationsymbol {\textcolor{red}{\llap{\tiny$\m@th\hookrightarrow$}}} 
        \newcommand*\Wrappedcontinuationindent {3ex } 
        \newcommand*\Wrappedafterbreak {\kern\Wrappedcontinuationindent\copy\Wrappedcontinuationbox} 
        % Take advantage of the already applied Pygments mark-up to insert 
        % potential linebreaks for TeX processing. 
        %        {, <, #, %, $, ' and ": go to next line. 
        %        _, }, ^, &, >, - and ~: stay at end of broken line. 
        % Use of \textquotesingle for straight quote. 
        \newcommand*\Wrappedbreaksatspecials {% 
            \def\PYGZus{\discretionary{\char`\_}{\Wrappedafterbreak}{\char`\_}}% 
            \def\PYGZob{\discretionary{}{\Wrappedafterbreak\char`\{}{\char`\{}}% 
            \def\PYGZcb{\discretionary{\char`\}}{\Wrappedafterbreak}{\char`\}}}% 
            \def\PYGZca{\discretionary{\char`\^}{\Wrappedafterbreak}{\char`\^}}% 
            \def\PYGZam{\discretionary{\char`\&}{\Wrappedafterbreak}{\char`\&}}% 
            \def\PYGZlt{\discretionary{}{\Wrappedafterbreak\char`\<}{\char`\<}}% 
            \def\PYGZgt{\discretionary{\char`\>}{\Wrappedafterbreak}{\char`\>}}% 
            \def\PYGZsh{\discretionary{}{\Wrappedafterbreak\char`\#}{\char`\#}}% 
            \def\PYGZpc{\discretionary{}{\Wrappedafterbreak\char`\%}{\char`\%}}% 
            \def\PYGZdl{\discretionary{}{\Wrappedafterbreak\char`\$}{\char`\$}}% 
            \def\PYGZhy{\discretionary{\char`\-}{\Wrappedafterbreak}{\char`\-}}% 
            \def\PYGZsq{\discretionary{}{\Wrappedafterbreak\textquotesingle}{\textquotesingle}}% 
            \def\PYGZdq{\discretionary{}{\Wrappedafterbreak\char`\"}{\char`\"}}% 
            \def\PYGZti{\discretionary{\char`\~}{\Wrappedafterbreak}{\char`\~}}% 
        } 
        % Some characters . , ; ? ! / are not pygmentized. 
        % This macro makes them "active" and they will insert potential linebreaks 
        \newcommand*\Wrappedbreaksatpunct {% 
            \lccode`\~`\.\lowercase{\def~}{\discretionary{\hbox{\char`\.}}{\Wrappedafterbreak}{\hbox{\char`\.}}}% 
            \lccode`\~`\,\lowercase{\def~}{\discretionary{\hbox{\char`\,}}{\Wrappedafterbreak}{\hbox{\char`\,}}}% 
            \lccode`\~`\;\lowercase{\def~}{\discretionary{\hbox{\char`\;}}{\Wrappedafterbreak}{\hbox{\char`\;}}}% 
            \lccode`\~`\:\lowercase{\def~}{\discretionary{\hbox{\char`\:}}{\Wrappedafterbreak}{\hbox{\char`\:}}}% 
            \lccode`\~`\?\lowercase{\def~}{\discretionary{\hbox{\char`\?}}{\Wrappedafterbreak}{\hbox{\char`\?}}}% 
            \lccode`\~`\!\lowercase{\def~}{\discretionary{\hbox{\char`\!}}{\Wrappedafterbreak}{\hbox{\char`\!}}}% 
            \lccode`\~`\/\lowercase{\def~}{\discretionary{\hbox{\char`\/}}{\Wrappedafterbreak}{\hbox{\char`\/}}}% 
            \catcode`\.\active
            \catcode`\,\active 
            \catcode`\;\active
            \catcode`\:\active
            \catcode`\?\active
            \catcode`\!\active
            \catcode`\/\active 
            \lccode`\~`\~ 	
        }
    \makeatother

    \let\OriginalVerbatim=\Verbatim
    \makeatletter
    \renewcommand{\Verbatim}[1][1]{%
        %\parskip\z@skip
        \sbox\Wrappedcontinuationbox {\Wrappedcontinuationsymbol}%
        \sbox\Wrappedvisiblespacebox {\FV@SetupFont\Wrappedvisiblespace}%
        \def\FancyVerbFormatLine ##1{\hsize\linewidth
            \vtop{\raggedright\hyphenpenalty\z@\exhyphenpenalty\z@
                \doublehyphendemerits\z@\finalhyphendemerits\z@
                \strut ##1\strut}%
        }%
        % If the linebreak is at a space, the latter will be displayed as visible
        % space at end of first line, and a continuation symbol starts next line.
        % Stretch/shrink are however usually zero for typewriter font.
        \def\FV@Space {%
            \nobreak\hskip\z@ plus\fontdimen3\font minus\fontdimen4\font
            \discretionary{\copy\Wrappedvisiblespacebox}{\Wrappedafterbreak}
            {\kern\fontdimen2\font}%
        }%
        
        % Allow breaks at special characters using \PYG... macros.
        \Wrappedbreaksatspecials
        % Breaks at punctuation characters . , ; ? ! and / need catcode=\active 	
        \OriginalVerbatim[#1,codes*=\Wrappedbreaksatpunct]%
    }
    \makeatother

    % Exact colors from NB
    \definecolor{incolor}{HTML}{303F9F}
    \definecolor{outcolor}{HTML}{D84315}
    \definecolor{cellborder}{HTML}{CFCFCF}
    \definecolor{cellbackground}{HTML}{F7F7F7}
    
    % prompt
    \makeatletter
    \newcommand{\boxspacing}{\kern\kvtcb@left@rule\kern\kvtcb@boxsep}
    \makeatother
    \newcommand{\prompt}[4]{
        \ttfamily\llap{{\color{#2}[#3]:\hspace{3pt}#4}}\vspace{-\baselineskip}
    }
    

    
    % Prevent overflowing lines due to hard-to-break entities
    \sloppy 
    % Setup hyperref package
    \hypersetup{
      breaklinks=true,  % so long urls are correctly broken across lines
      colorlinks=true,
      urlcolor=urlcolor,
      linkcolor=linkcolor,
      citecolor=citecolor,
      }
    % Slightly bigger margins than the latex defaults
    
    \geometry{verbose,tmargin=1in,bmargin=1in,lmargin=1in,rmargin=1in}
    
    

\begin{document}
    
    \maketitle
    
    

    
    \begin{tcolorbox}[breakable, size=fbox, boxrule=1pt, pad at break*=1mm,colback=cellbackground, colframe=cellborder]
\prompt{In}{incolor}{200}{\boxspacing}
\begin{Verbatim}[commandchars=\\\{\}]
\PY{k+kn}{import} \PY{n+nn}{numpy} \PY{k}{as} \PY{n+nn}{np}
\PY{k+kn}{from} \PY{n+nn}{astropy}\PY{n+nn}{.}\PY{n+nn}{time} \PY{k+kn}{import} \PY{n}{Time}\PY{p}{,} \PY{n}{TimeDelta}
\PY{k+kn}{from} \PY{n+nn}{astropy}\PY{n+nn}{.}\PY{n+nn}{coordinates} \PY{k+kn}{import} \PY{n}{EarthLocation}
\PY{k+kn}{from} \PY{n+nn}{astropy} \PY{k+kn}{import} \PY{n}{units} \PY{k}{as} \PY{n}{u}
\PY{k+kn}{import} \PY{n+nn}{pandas} \PY{k}{as} \PY{n+nn}{pd}
\PY{k+kn}{from} \PY{n+nn}{astroplan} \PY{k+kn}{import} \PY{p}{(}\PY{n}{FixedTarget}\PY{p}{,} \PY{n}{Observer}\PY{p}{,} 
                       \PY{n}{AltitudeConstraint}\PY{p}{,} \PY{n}{AtNightConstraint}\PY{p}{,} 
                       \PY{n}{MoonSeparationConstraint}\PY{p}{)}
\PY{k+kn}{import} \PY{n+nn}{matplotlib}\PY{n+nn}{.}\PY{n+nn}{pyplot} \PY{k}{as} \PY{n+nn}{plt}
\PY{k+kn}{from} \PY{n+nn}{matplotlib}\PY{n+nn}{.}\PY{n+nn}{ticker} \PY{k+kn}{import} \PY{p}{(}\PY{n}{MultipleLocator}\PY{p}{,} \PY{n}{FormatStrFormatter}\PY{p}{,}
                               \PY{n}{AutoMinorLocator}\PY{p}{)}
\PY{k+kn}{from} \PY{n+nn}{datetime} \PY{k+kn}{import} \PY{n}{datetime}
\PY{n}{plt}\PY{o}{.}\PY{n}{rcParams}\PY{o}{.}\PY{n}{update}\PY{p}{(}\PY{p}{\PYZob{}}\PY{l+s+s1}{\PYZsq{}}\PY{l+s+s1}{font.size}\PY{l+s+s1}{\PYZsq{}}\PY{p}{:} \PY{l+m+mi}{22}\PY{p}{\PYZcb{}}\PY{p}{)}
\end{Verbatim}
\end{tcolorbox}

    On définit une fonction contraintes, basé sur les librairies astropy et
astroplan, qui nous permet de connaître les dates auxquelles il est
possible d'observer une certaine phase de notre étoile en fonction de
contraintes que l'on souhaite imposer :

\begin{verbatim}
-Il fait nuit    
-L'étoile est à un angle supérieur à 30° de l'horizon    
-L'étoile est à un angle supérieur à 30° de la lune
\end{verbatim}

On entre les coordonnées de l'observatoire (New mexico skies), ainsi que
le nom de l'étoile que l'on souhaite observer, puis l'on crée un tableau
au format Time qui contient les `amount' prochaines dates auxquelles
l'étoile est aux différentes phases demandées. On vérifie en même temps
que les contraintes soient respectées. Si les contraintes sont
respectées, la date est conservée. Sinon, elle est supprimée. On renvoit
un tableau au format DataFrame qui contient des colonnes correspondant
aux différentes phases que l'on souhaite observer, et pour lesquelles on
a une succession de dates d'observation possible.

    \begin{tcolorbox}[breakable, size=fbox, boxrule=1pt, pad at break*=1mm,colback=cellbackground, colframe=cellborder]
\prompt{In}{incolor}{196}{\boxspacing}
\begin{Verbatim}[commandchars=\\\{\}]
\PY{k}{def} \PY{n+nf}{constraintes}\PY{p}{(}\PY{n}{amount}\PY{p}{,} \PY{n}{minimum}\PY{p}{,} \PY{n}{periode}\PY{p}{,} \PY{n}{phases}\PY{p}{)} \PY{p}{:}
    
    \PY{c+c1}{\PYZsh{}On définit les données relatives à l\PYZsq{}observatoire et à notre étoile}
    \PY{n}{T5} \PY{o}{=} \PY{n}{EarthLocation}\PY{p}{(}\PY{n}{lat} \PY{o}{=} \PY{l+m+mf}{32.9} \PY{o}{*} \PY{n}{u}\PY{o}{.}\PY{n}{deg}\PY{p}{,} \PY{n}{lon} \PY{o}{=} \PY{o}{\PYZhy{}}\PY{l+m+mf}{105.5} \PY{o}{*} \PY{n}{u}\PY{o}{.}\PY{n}{deg}\PY{p}{,} \PY{n}{height} \PY{o}{=} \PY{l+m+mi}{2225} \PY{o}{*} \PY{n}{u}\PY{o}{.}\PY{n}{m}\PY{p}{)}
    \PY{n}{observatoire} \PY{o}{=} \PY{n}{Observer}\PY{p}{(}\PY{n}{location} \PY{o}{=} \PY{n}{T5}\PY{p}{,} \PY{n}{name} \PY{o}{=} \PY{l+s+s2}{\PYZdq{}}\PY{l+s+s2}{T5}\PY{l+s+s2}{\PYZdq{}}\PY{p}{)}
    \PY{n}{target} \PY{o}{=} \PY{n}{FixedTarget}\PY{o}{.}\PY{n}{from\PYZus{}name}\PY{p}{(}\PY{l+s+s2}{\PYZdq{}}\PY{l+s+s2}{DU BOO}\PY{l+s+s2}{\PYZdq{}}\PY{p}{)}
    
    \PY{c+c1}{\PYZsh{}On définit nos contraintes}
    \PY{n}{constraints} \PY{o}{=} \PY{p}{[}\PY{n}{AtNightConstraint}\PY{o}{.}\PY{n}{twilight\PYZus{}astronomical}\PY{p}{(}\PY{p}{)}\PY{p}{,}
                  \PY{n}{MoonSeparationConstraint}\PY{p}{(}\PY{n+nb}{min} \PY{o}{=} \PY{l+m+mi}{30} \PY{o}{*}\PY{n}{u}\PY{o}{.}\PY{n}{deg}\PY{p}{)}\PY{p}{,}
                  \PY{n}{AltitudeConstraint}\PY{p}{(}\PY{n+nb}{min} \PY{o}{=} \PY{l+m+mi}{30}\PY{o}{*} \PY{n}{u}\PY{o}{.}\PY{n}{deg}\PY{p}{)}\PY{p}{]}
    
    \PY{c+c1}{\PYZsh{}On convertit les entrée au format Time}
    \PY{n}{start\PYZus{}min} \PY{o}{=} \PY{n}{Time}\PY{p}{(}\PY{n}{minimum}\PY{p}{,} \PY{n+nb}{format} \PY{o}{=} \PY{l+s+s1}{\PYZsq{}}\PY{l+s+s1}{jd}\PY{l+s+s1}{\PYZsq{}}\PY{p}{)}
    \PY{n}{period} \PY{o}{=} \PY{n}{TimeDelta}\PY{p}{(}\PY{n}{periode}\PY{p}{,} \PY{n+nb}{format} \PY{o}{=} \PY{l+s+s1}{\PYZsq{}}\PY{l+s+s1}{jd}\PY{l+s+s1}{\PYZsq{}}\PY{p}{)}
    
    \PY{c+c1}{\PYZsh{}On crée un tableau vide au format Time, et un tableau correspondant qui contiendra l\PYZsq{}information}
    \PY{c+c1}{\PYZsh{}sur les contraintes}
    \PY{n}{time} \PY{o}{=} \PY{n}{Time}\PY{p}{(}\PY{n}{np}\PY{o}{.}\PY{n}{ones}\PY{p}{(}\PY{p}{(}\PY{n}{amount}\PY{p}{,} \PY{n+nb}{len}\PY{p}{(}\PY{n}{phases}\PY{p}{)}\PY{p}{)}\PY{p}{)}\PY{p}{,} \PY{n+nb}{format} \PY{o}{=} \PY{l+s+s1}{\PYZsq{}}\PY{l+s+s1}{jd}\PY{l+s+s1}{\PYZsq{}}\PY{p}{)}
    \PY{n}{ok} \PY{o}{=} \PY{n}{np}\PY{o}{.}\PY{n}{zeros\PYZus{}like}\PY{p}{(}\PY{n}{time}\PY{p}{)}
    
    \PY{c+c1}{\PYZsh{}On vérifie les contraintes}
    \PY{k}{for} \PY{n}{n} \PY{o+ow}{in} \PY{n+nb}{range} \PY{p}{(}\PY{n}{amount}\PY{p}{)} \PY{p}{:}
        \PY{n}{time}\PY{p}{[}\PY{n}{n}\PY{p}{]} \PY{o}{=} \PY{n}{start\PYZus{}min} \PY{o}{+} \PY{n}{periode} \PY{o}{*} \PY{p}{(}\PY{n}{phases} \PY{o}{+} \PY{n}{n}\PY{p}{)}
        
        \PY{n}{ok}\PY{p}{[}\PY{n}{n}\PY{p}{]} \PY{o}{=} \PY{n}{np}\PY{o}{.}\PY{n}{logical\PYZus{}and}\PY{p}{(}\PY{n}{constraints}\PY{p}{[}\PY{l+m+mi}{0}\PY{p}{]}\PY{p}{(}\PY{n}{observatoire}\PY{p}{,} \PY{n}{target}\PY{p}{,} \PY{n}{times} \PY{o}{=} \PY{n}{time}\PY{p}{[}\PY{n}{n}\PY{p}{]}\PY{p}{)} \PY{o}{==} \PY{k+kc}{True}\PY{p}{,} 
                               \PY{n}{np}\PY{o}{.}\PY{n}{logical\PYZus{}and}\PY{p}{(}\PY{n}{constraints}\PY{p}{[}\PY{l+m+mi}{1}\PY{p}{]}\PY{p}{(}\PY{n}{observatoire}\PY{p}{,} \PY{n}{target}\PY{p}{,} \PY{n}{times} \PY{o}{=} \PY{n}{time}\PY{p}{[}\PY{n}{n}\PY{p}{]}\PY{p}{)} \PY{o}{==} \PY{k+kc}{True}\PY{p}{,}
                                              \PY{n}{constraints}\PY{p}{[}\PY{l+m+mi}{2}\PY{p}{]}\PY{p}{(}\PY{n}{observatoire}\PY{p}{,} \PY{n}{target}\PY{p}{,} \PY{n}{times} \PY{o}{=} \PY{n}{time}\PY{p}{[}\PY{n}{n}\PY{p}{]}\PY{p}{)} \PY{o}{==} \PY{k+kc}{True}\PY{p}{)}\PY{p}{)}
    
    \PY{c+c1}{\PYZsh{}On se replace à l\PYZsq{}heure = UTC \PYZhy{} 7}
    \PY{n}{time} \PY{o}{=} \PY{n}{time} \PY{o}{\PYZhy{}} \PY{l+m+mi}{7} \PY{o}{*} \PY{n}{u}\PY{o}{.}\PY{n}{hour}
    
    \PY{c+c1}{\PYZsh{}On extrait un array de notre tableau time, au format de dates iso}
    \PY{n}{time\PYZus{}obs} \PY{o}{=} \PY{n}{time}\PY{o}{.}\PY{n}{iso}
    
    \PY{c+c1}{\PYZsh{}On transforme l\PYZsq{}array en un dataFrame}
    \PY{n}{df} \PY{o}{=} \PY{n}{pd}\PY{o}{.}\PY{n}{DataFrame}\PY{p}{(}\PY{n}{data}\PY{o}{=}\PY{n}{time\PYZus{}obs}\PY{p}{,} \PY{n}{columns}\PY{o}{=}\PY{n}{phases}\PY{p}{)}
    
    \PY{c+c1}{\PYZsh{}On supprime les entrées la où les contraintes ne sont pas respectées    }
    \PY{n}{df} \PY{o}{=} \PY{n}{df}\PY{o}{.}\PY{n}{where}\PY{p}{(}\PY{n}{ok} \PY{o}{!=} \PY{k+kc}{False}\PY{p}{)}
    
    \PY{k}{return} \PY{n}{df}
    
\end{Verbatim}
\end{tcolorbox}

    Afin de déterminer les phases que l'on souhaite observer, on utilise les
données disponible sur
\href{'http://cdsarc.u-strasbg.fr/viz-bin/vizExec/Vgraph?I/239/70240\&P=1.0558882\&P=1.0558882'}{le
site de l'université de strasbourg} On trace un graphique de ces
données, que l'on décale afin de placer le début de la période au milieu
du minimum principal de luminosité. On place ensuite les 15 points
d'observations, d'abord en séparant uniformément la période, puis en
ajustant afin d'avoir des points coincidant avec les maximas et minimas.
On a alors la liste des phases que l'on souhaite observer

    \begin{tcolorbox}[breakable, size=fbox, boxrule=1pt, pad at break*=1mm,colback=cellbackground, colframe=cellborder]
\prompt{In}{incolor}{51}{\boxspacing}
\begin{Verbatim}[commandchars=\\\{\}]
\PY{n}{ephe} \PY{o}{=} \PY{n}{pd}\PY{o}{.}\PY{n}{read\PYZus{}csv}\PY{p}{(}\PY{l+s+s1}{\PYZsq{}}\PY{l+s+s1}{data/luminosite}\PY{l+s+s1}{\PYZsq{}}\PY{p}{,} \PY{n}{delimiter} \PY{o}{=} \PY{l+s+s1}{\PYZsq{}}\PY{l+s+s1}{ }\PY{l+s+s1}{\PYZsq{}}\PY{p}{,} \PY{n}{skiprows} \PY{o}{=} \PY{l+m+mi}{6}\PY{p}{,} \PY{n}{header} \PY{o}{=} \PY{k+kc}{None}\PY{p}{,}
                   \PY{n}{names} \PY{o}{=} \PY{p}{[}\PY{l+s+s1}{\PYZsq{}}\PY{l+s+s1}{phase}\PY{l+s+s1}{\PYZsq{}}\PY{p}{,} \PY{l+s+s1}{\PYZsq{}}\PY{l+s+s1}{\PYZsq{}}\PY{p}{,} \PY{l+s+s1}{\PYZsq{}}\PY{l+s+s1}{luminosite}\PY{l+s+s1}{\PYZsq{}}\PY{p}{,} \PY{l+s+s1}{\PYZsq{}}\PY{l+s+s1}{erreur}\PY{l+s+s1}{\PYZsq{}}\PY{p}{]}\PY{p}{)}
\PY{n}{lum} \PY{o}{=} \PY{n}{ephe}\PY{o}{.}\PY{n}{copy}\PY{p}{(}\PY{p}{)}
\PY{n}{lum}\PY{p}{[}\PY{l+s+s1}{\PYZsq{}}\PY{l+s+s1}{phase}\PY{l+s+s1}{\PYZsq{}}\PY{p}{]} \PY{o}{=} \PY{n}{lum}\PY{p}{[}\PY{l+s+s1}{\PYZsq{}}\PY{l+s+s1}{phase}\PY{l+s+s1}{\PYZsq{}}\PY{p}{]} \PY{o}{+}\PY{l+m+mi}{1}
\PY{n}{lum}\PY{o}{.}\PY{n}{head}\PY{p}{(}\PY{p}{)}
\end{Verbatim}
\end{tcolorbox}

            \begin{tcolorbox}[breakable, size=fbox, boxrule=.5pt, pad at break*=1mm, opacityfill=0]
\prompt{Out}{outcolor}{51}{\boxspacing}
\begin{Verbatim}[commandchars=\\\{\}]
      phase      luminosite  erreur
0  1.239611 NaN      8.7616   0.018
1  1.582669 NaN      8.6570   0.017
2  1.596250 NaN      8.6454   0.017
3  1.666863 NaN      8.5992   0.015
4  1.279503 NaN      8.7985   0.012
\end{Verbatim}
\end{tcolorbox}
        
    \begin{tcolorbox}[breakable, size=fbox, boxrule=1pt, pad at break*=1mm,colback=cellbackground, colframe=cellborder]
\prompt{In}{incolor}{167}{\boxspacing}
\begin{Verbatim}[commandchars=\\\{\}]
\PY{n}{fig}\PY{p}{,} \PY{n}{ax} \PY{o}{=} \PY{n}{plt}\PY{o}{.}\PY{n}{subplots}\PY{p}{(}\PY{l+m+mi}{1}\PY{p}{,} \PY{l+m+mi}{2}\PY{p}{,} \PY{n}{dpi} \PY{o}{=} \PY{l+m+mi}{300}\PY{p}{,} \PY{n}{figsize} \PY{o}{=} \PY{p}{(}\PY{l+m+mi}{16}\PY{p}{,} \PY{l+m+mi}{9}\PY{p}{)}\PY{p}{,} \PY{n}{sharey} \PY{o}{=} \PY{k+kc}{True}\PY{p}{)}

\PY{n}{ax}\PY{p}{[}\PY{l+m+mi}{0}\PY{p}{]}\PY{o}{.}\PY{n}{set\PYZus{}title}\PY{p}{(}\PY{l+s+s1}{\PYZsq{}}\PY{l+s+s1}{(a)}\PY{l+s+se}{\PYZbs{}n}\PY{l+s+s1}{\PYZsq{}}\PY{p}{)}

\PY{n}{ax}\PY{p}{[}\PY{l+m+mi}{0}\PY{p}{]}\PY{o}{.}\PY{n}{errorbar}\PY{p}{(}\PY{n}{ephe}\PY{p}{[}\PY{l+s+s1}{\PYZsq{}}\PY{l+s+s1}{phase}\PY{l+s+s1}{\PYZsq{}}\PY{p}{]}\PY{p}{,} \PY{n}{ephe}\PY{p}{[}\PY{l+s+s1}{\PYZsq{}}\PY{l+s+s1}{luminosite}\PY{l+s+s1}{\PYZsq{}}\PY{p}{]}\PY{p}{,} \PY{n}{fmt} \PY{o}{=} \PY{l+s+s1}{\PYZsq{}}\PY{l+s+s1}{\PYZca{}}\PY{l+s+s1}{\PYZsq{}}\PY{p}{,} \PY{n}{yerr} \PY{o}{=} \PY{n}{ephe}\PY{p}{[}\PY{l+s+s1}{\PYZsq{}}\PY{l+s+s1}{erreur}\PY{l+s+s1}{\PYZsq{}}\PY{p}{]}\PY{p}{,}
            \PY{n}{capsize} \PY{o}{=} \PY{l+m+mi}{5}\PY{p}{,} \PY{n}{c} \PY{o}{=} \PY{l+s+s1}{\PYZsq{}}\PY{l+s+s1}{orange}\PY{l+s+s1}{\PYZsq{}}\PY{p}{)}

\PY{n}{ax}\PY{p}{[}\PY{l+m+mi}{0}\PY{p}{]}\PY{o}{.}\PY{n}{xaxis}\PY{o}{.}\PY{n}{set\PYZus{}major\PYZus{}locator}\PY{p}{(}\PY{n}{MultipleLocator}\PY{p}{(}\PY{l+m+mf}{0.1}\PY{p}{)}\PY{p}{)}
\PY{n}{ax}\PY{p}{[}\PY{l+m+mi}{0}\PY{p}{]}\PY{o}{.}\PY{n}{grid}\PY{p}{(}\PY{n}{which} \PY{o}{=} \PY{l+s+s1}{\PYZsq{}}\PY{l+s+s1}{both}\PY{l+s+s1}{\PYZsq{}}\PY{p}{)}

\PY{n}{ax}\PY{p}{[}\PY{l+m+mi}{0}\PY{p}{]}\PY{o}{.}\PY{n}{set\PYZus{}xlim}\PY{p}{(}\PY{l+m+mi}{0}\PY{p}{,} \PY{l+m+mi}{1}\PY{p}{)}
\PY{n}{ax}\PY{p}{[}\PY{l+m+mi}{0}\PY{p}{]}\PY{o}{.}\PY{n}{set\PYZus{}xlabel}\PY{p}{(}\PY{l+s+s1}{\PYZsq{}}\PY{l+s+s1}{Phase}\PY{l+s+s1}{\PYZsq{}}\PY{p}{)}

\PY{n}{ax}\PY{p}{[}\PY{l+m+mi}{0}\PY{p}{]}\PY{o}{.}\PY{n}{set\PYZus{}ylim}\PY{p}{(}\PY{l+m+mf}{8.5}\PY{p}{,} \PY{l+m+mf}{9.3}\PY{p}{)}
\PY{n}{ax}\PY{p}{[}\PY{l+m+mi}{0}\PY{p}{]}\PY{o}{.}\PY{n}{set\PYZus{}ylabel}\PY{p}{(}\PY{l+s+s1}{\PYZsq{}}\PY{l+s+s1}{Magnitude}\PY{l+s+s1}{\PYZsq{}}\PY{p}{)}
\PY{n}{ax}\PY{p}{[}\PY{l+m+mi}{0}\PY{p}{]}\PY{o}{.}\PY{n}{invert\PYZus{}yaxis}\PY{p}{(}\PY{p}{)}


\PY{n}{ax}\PY{p}{[}\PY{l+m+mi}{1}\PY{p}{]}\PY{o}{.}\PY{n}{set\PYZus{}title}\PY{p}{(}\PY{l+s+s1}{\PYZsq{}}\PY{l+s+s1}{(b)}\PY{l+s+se}{\PYZbs{}n}\PY{l+s+s1}{\PYZsq{}}\PY{p}{)}

\PY{n}{ax}\PY{p}{[}\PY{l+m+mi}{1}\PY{p}{]}\PY{o}{.}\PY{n}{errorbar}\PY{p}{(}\PY{n}{lum}\PY{p}{[}\PY{l+s+s1}{\PYZsq{}}\PY{l+s+s1}{phase}\PY{l+s+s1}{\PYZsq{}}\PY{p}{]} \PY{o}{\PYZhy{}} \PY{l+m+mf}{0.395}\PY{p}{,} \PY{n}{lum}\PY{p}{[}\PY{l+s+s1}{\PYZsq{}}\PY{l+s+s1}{luminosite}\PY{l+s+s1}{\PYZsq{}}\PY{p}{]}\PY{p}{,} \PY{n}{fmt} \PY{o}{=} \PY{l+s+s1}{\PYZsq{}}\PY{l+s+s1}{\PYZca{}}\PY{l+s+s1}{\PYZsq{}}\PY{p}{,} \PY{n}{yerr} \PY{o}{=} \PY{n}{lum}\PY{p}{[}\PY{l+s+s1}{\PYZsq{}}\PY{l+s+s1}{erreur}\PY{l+s+s1}{\PYZsq{}}\PY{p}{]}\PY{p}{,}
            \PY{n}{capsize} \PY{o}{=} \PY{l+m+mi}{5}\PY{p}{,}  \PY{n}{c} \PY{o}{=}\PY{l+s+s1}{\PYZsq{}}\PY{l+s+s1}{orange}\PY{l+s+s1}{\PYZsq{}}\PY{p}{)}
\PY{n}{ax}\PY{p}{[}\PY{l+m+mi}{1}\PY{p}{]}\PY{o}{.}\PY{n}{errorbar}\PY{p}{(}\PY{n}{ephe}\PY{p}{[}\PY{l+s+s1}{\PYZsq{}}\PY{l+s+s1}{phase}\PY{l+s+s1}{\PYZsq{}}\PY{p}{]} \PY{o}{\PYZhy{}} \PY{l+m+mf}{0.395}\PY{p}{,} \PY{n}{ephe}\PY{p}{[}\PY{l+s+s1}{\PYZsq{}}\PY{l+s+s1}{luminosite}\PY{l+s+s1}{\PYZsq{}}\PY{p}{]}\PY{p}{,} \PY{n}{fmt} \PY{o}{=} \PY{l+s+s1}{\PYZsq{}}\PY{l+s+s1}{\PYZca{}}\PY{l+s+s1}{\PYZsq{}}\PY{p}{,} \PY{n}{yerr} \PY{o}{=} \PY{n}{ephe}\PY{p}{[}\PY{l+s+s1}{\PYZsq{}}\PY{l+s+s1}{erreur}\PY{l+s+s1}{\PYZsq{}}\PY{p}{]}\PY{p}{,}
            \PY{n}{capsize} \PY{o}{=} \PY{l+m+mi}{5}\PY{p}{,} \PY{n}{c} \PY{o}{=} \PY{l+s+s1}{\PYZsq{}}\PY{l+s+s1}{orange}\PY{l+s+s1}{\PYZsq{}}\PY{p}{)}

\PY{n}{ax}\PY{p}{[}\PY{l+m+mi}{1}\PY{p}{]}\PY{o}{.}\PY{n}{xaxis}\PY{o}{.}\PY{n}{set\PYZus{}major\PYZus{}locator}\PY{p}{(}\PY{n}{MultipleLocator}\PY{p}{(}\PY{l+m+mf}{0.1}\PY{p}{)}\PY{p}{)}
\PY{n}{ax}\PY{p}{[}\PY{l+m+mi}{1}\PY{p}{]}\PY{o}{.}\PY{n}{grid}\PY{p}{(}\PY{n}{which} \PY{o}{=} \PY{l+s+s1}{\PYZsq{}}\PY{l+s+s1}{both}\PY{l+s+s1}{\PYZsq{}}\PY{p}{)}

\PY{n}{ax}\PY{p}{[}\PY{l+m+mi}{1}\PY{p}{]}\PY{o}{.}\PY{n}{set\PYZus{}xlim}\PY{p}{(}\PY{l+m+mi}{0}\PY{p}{,} \PY{l+m+mi}{1}\PY{p}{)}
\PY{n}{ax}\PY{p}{[}\PY{l+m+mi}{1}\PY{p}{]}\PY{o}{.}\PY{n}{set\PYZus{}xlabel}\PY{p}{(}\PY{l+s+s1}{\PYZsq{}}\PY{l+s+s1}{Phase}\PY{l+s+s1}{\PYZsq{}}\PY{p}{)}

\PY{n}{ax}\PY{p}{[}\PY{l+m+mi}{1}\PY{p}{]}\PY{o}{.}\PY{n}{set\PYZus{}ylim}\PY{p}{(}\PY{l+m+mf}{8.5}\PY{p}{,} \PY{l+m+mf}{9.3}\PY{p}{)}
\PY{n}{ax}\PY{p}{[}\PY{l+m+mi}{1}\PY{p}{]}\PY{o}{.}\PY{n}{invert\PYZus{}yaxis}\PY{p}{(}\PY{p}{)}

\PY{n}{fig}\PY{o}{.}\PY{n}{tight\PYZus{}layout}\PY{p}{(}\PY{p}{)}

\PY{n}{fig}\PY{o}{.}\PY{n}{savefig}\PY{p}{(}\PY{l+s+s1}{\PYZsq{}}\PY{l+s+s1}{figures/cds\PYZus{}luminosite\PYZus{}centree.png}\PY{l+s+s1}{\PYZsq{}}\PY{p}{)}
\end{Verbatim}
\end{tcolorbox}

    \begin{center}
    \adjustimage{max size={0.9\linewidth}{0.9\paperheight}}{Calcul dates d'observation du boo_files/Calcul dates d'observation du boo_5_0.png}
    \end{center}
    { \hspace*{\fill} \\}
    
    \begin{tcolorbox}[breakable, size=fbox, boxrule=1pt, pad at break*=1mm,colback=cellbackground, colframe=cellborder]
\prompt{In}{incolor}{156}{\boxspacing}
\begin{Verbatim}[commandchars=\\\{\}]
\PY{n}{phases} \PY{o}{=} \PY{n}{np}\PY{o}{.}\PY{n}{linspace}\PY{p}{(}\PY{l+m+mi}{0}\PY{p}{,} \PY{l+m+mi}{1}\PY{p}{,} \PY{l+m+mi}{16}\PY{p}{)}
\PY{n}{phases}\PY{p}{[}\PY{o}{\PYZhy{}}\PY{l+m+mi}{8}\PY{p}{:}\PY{p}{]} \PY{o}{=} \PY{p}{[}\PY{l+m+mf}{0.52} \PY{p}{,} \PY{l+m+mf}{0.59}\PY{p}{,} \PY{l+m+mf}{0.67}\PY{p}{,}\PY{l+m+mf}{0.76}\PY{p}{,} \PY{l+m+mf}{0.83}\PY{p}{,} \PY{l+m+mf}{0.91}\PY{p}{,} \PY{l+m+mf}{0.98}\PY{p}{,} \PY{l+m+mf}{1.0}\PY{p}{]}


\PY{n}{fig}\PY{p}{,} \PY{n}{ax} \PY{o}{=} \PY{n}{plt}\PY{o}{.}\PY{n}{subplots}\PY{p}{(}\PY{l+m+mi}{1}\PY{p}{,} \PY{l+m+mi}{1}\PY{p}{,} \PY{n}{dpi} \PY{o}{=} \PY{l+m+mi}{300}\PY{p}{,} \PY{n}{figsize} \PY{o}{=} \PY{p}{(}\PY{l+m+mi}{16}\PY{p}{,} \PY{l+m+mi}{9}\PY{p}{)}\PY{p}{)}




\PY{n}{ax}\PY{o}{.}\PY{n}{errorbar}\PY{p}{(}\PY{n}{lum}\PY{p}{[}\PY{l+s+s1}{\PYZsq{}}\PY{l+s+s1}{phase}\PY{l+s+s1}{\PYZsq{}}\PY{p}{]} \PY{o}{\PYZhy{}} \PY{l+m+mf}{0.395}\PY{p}{,} \PY{n}{lum}\PY{p}{[}\PY{l+s+s1}{\PYZsq{}}\PY{l+s+s1}{luminosite}\PY{l+s+s1}{\PYZsq{}}\PY{p}{]}\PY{p}{,} \PY{n}{fmt} \PY{o}{=} \PY{l+s+s1}{\PYZsq{}}\PY{l+s+s1}{\PYZca{}}\PY{l+s+s1}{\PYZsq{}}\PY{p}{,} \PY{n}{yerr} \PY{o}{=} \PY{n}{lum}\PY{p}{[}\PY{l+s+s1}{\PYZsq{}}\PY{l+s+s1}{erreur}\PY{l+s+s1}{\PYZsq{}}\PY{p}{]}\PY{p}{,}
            \PY{n}{capsize} \PY{o}{=} \PY{l+m+mi}{5}\PY{p}{,}  \PY{n}{c} \PY{o}{=}\PY{l+s+s1}{\PYZsq{}}\PY{l+s+s1}{orange}\PY{l+s+s1}{\PYZsq{}}\PY{p}{)}
\PY{n}{ax}\PY{o}{.}\PY{n}{errorbar}\PY{p}{(}\PY{n}{ephe}\PY{p}{[}\PY{l+s+s1}{\PYZsq{}}\PY{l+s+s1}{phase}\PY{l+s+s1}{\PYZsq{}}\PY{p}{]} \PY{o}{\PYZhy{}} \PY{l+m+mf}{0.395}\PY{p}{,} \PY{n}{ephe}\PY{p}{[}\PY{l+s+s1}{\PYZsq{}}\PY{l+s+s1}{luminosite}\PY{l+s+s1}{\PYZsq{}}\PY{p}{]}\PY{p}{,} \PY{n}{fmt} \PY{o}{=} \PY{l+s+s1}{\PYZsq{}}\PY{l+s+s1}{\PYZca{}}\PY{l+s+s1}{\PYZsq{}}\PY{p}{,} \PY{n}{yerr} \PY{o}{=} \PY{n}{ephe}\PY{p}{[}\PY{l+s+s1}{\PYZsq{}}\PY{l+s+s1}{erreur}\PY{l+s+s1}{\PYZsq{}}\PY{p}{]}\PY{p}{,}
            \PY{n}{capsize} \PY{o}{=} \PY{l+m+mi}{5}\PY{p}{,} \PY{n}{c} \PY{o}{=} \PY{l+s+s1}{\PYZsq{}}\PY{l+s+s1}{orange}\PY{l+s+s1}{\PYZsq{}}\PY{p}{)}
\PY{n}{ax}\PY{o}{.}\PY{n}{vlines}\PY{p}{(}\PY{n}{phases}\PY{p}{,} \PY{l+m+mf}{8.4}\PY{p}{,} \PY{l+m+mf}{9.5}\PY{p}{,} \PY{n}{color} \PY{o}{=} \PY{l+s+s1}{\PYZsq{}}\PY{l+s+s1}{g}\PY{l+s+s1}{\PYZsq{}}\PY{p}{,} \PY{n}{label} \PY{o}{=} \PY{l+s+s1}{\PYZsq{}}\PY{l+s+s1}{Phases d}\PY{l+s+se}{\PYZbs{}\PYZsq{}}\PY{l+s+s1}{observation}\PY{l+s+s1}{\PYZsq{}}\PY{p}{)}




\PY{n}{ax}\PY{o}{.}\PY{n}{set\PYZus{}xlim}\PY{p}{(}\PY{l+m+mi}{0}\PY{p}{,} \PY{l+m+mi}{1}\PY{p}{)}
\PY{n}{ax}\PY{o}{.}\PY{n}{set\PYZus{}xlabel}\PY{p}{(}\PY{l+s+s1}{\PYZsq{}}\PY{l+s+s1}{Phase}\PY{l+s+s1}{\PYZsq{}}\PY{p}{)}

\PY{n}{ax}\PY{o}{.}\PY{n}{set\PYZus{}ylim}\PY{p}{(}\PY{l+m+mf}{8.4}\PY{p}{,} \PY{l+m+mf}{9.5}\PY{p}{)}
\PY{n}{ax}\PY{o}{.}\PY{n}{set\PYZus{}ylabel}\PY{p}{(}\PY{l+s+s1}{\PYZsq{}}\PY{l+s+s1}{Magnitude}\PY{l+s+s1}{\PYZsq{}}\PY{p}{)}
\PY{n}{ax}\PY{o}{.}\PY{n}{invert\PYZus{}yaxis}\PY{p}{(}\PY{p}{)}

\PY{n}{ax}\PY{o}{.}\PY{n}{set\PYZus{}xticks}\PY{p}{(}\PY{n}{np}\PY{o}{.}\PY{n}{around}\PY{p}{(}\PY{n}{phases}\PY{p}{[}\PY{l+m+mi}{0}\PY{p}{:}\PY{o}{\PYZhy{}}\PY{l+m+mi}{1}\PY{p}{]}\PY{p}{,} \PY{l+m+mi}{2}\PY{p}{)}\PY{p}{)}
\PY{n}{ax}\PY{o}{.}\PY{n}{grid}\PY{p}{(}\PY{p}{)}

\PY{n}{ax}\PY{o}{.}\PY{n}{legend}\PY{p}{(}\PY{p}{)}

\PY{n}{fig}\PY{o}{.}\PY{n}{tight\PYZus{}layout}\PY{p}{(}\PY{p}{)}

\PY{n}{fig}\PY{o}{.}\PY{n}{savefig}\PY{p}{(}\PY{l+s+s1}{\PYZsq{}}\PY{l+s+s1}{figures/cds\PYZus{}luminosite\PYZus{}centree\PYZus{}phases}\PY{l+s+s1}{\PYZsq{}}\PY{p}{)}
\end{Verbatim}
\end{tcolorbox}

    \begin{center}
    \adjustimage{max size={0.9\linewidth}{0.9\paperheight}}{Calcul dates d'observation du boo_files/Calcul dates d'observation du boo_6_0.png}
    \end{center}
    { \hspace*{\fill} \\}
    
    On initialise une variable contenant la période de l'étoile variable,
ainsi qu'une variable contenant la date en jour julien correspondant au
milieu de son minimum de luminosité, d'après l'éphéméride
\href{'https://www.aavso.org/vsx/index.php?view=detail.ephemeris\&nolayout=1\&oid=4446'}{disponible
ici} De plus, on crée un array contenant les différentes phases de
l'étoile que l'on souhaite observer. Il faut noter que le début de la
période est ici choisie en le minimum de luminosité de l'étoile.

On envoie alors ces données dans notre fonction contraintes(), qui nous
renvoie un tableau au format DataFrame, contenant une liste de dates
pour lesquelles les contraintes que l'on a choisies sont respectées.
Attention les dates sont à l'heure locale ! UTC - 7

    \begin{tcolorbox}[breakable, size=fbox, boxrule=1pt, pad at break*=1mm,colback=cellbackground, colframe=cellborder]
\prompt{In}{incolor}{201}{\boxspacing}
\begin{Verbatim}[commandchars=\\\{\}]
\PY{n}{minimum} \PY{o}{=} \PY{l+m+mf}{2458893.438}
\PY{n}{periode} \PY{o}{=} \PY{l+m+mf}{1.0558882}

\PY{c+c1}{\PYZsh{}On avance dans le temps jusqu\PYZsq{}à ce que la date indiqué pour le minimum soit postérieure à la date}
\PY{c+c1}{\PYZsh{}actuelle}
\PY{n}{today} \PY{o}{=} \PY{n}{Time}\PY{p}{(}\PY{n}{datetime}\PY{o}{.}\PY{n}{utcnow}\PY{p}{(}\PY{p}{)}\PY{p}{,} \PY{n+nb}{format} \PY{o}{=} \PY{l+s+s1}{\PYZsq{}}\PY{l+s+s1}{datetime}\PY{l+s+s1}{\PYZsq{}}\PY{p}{)}
\PY{k}{while} \PY{n}{minimum} \PY{o}{\PYZlt{}} \PY{n}{today}\PY{o}{.}\PY{n}{jd} \PY{p}{:}
    \PY{n}{minimum} \PY{o}{=} \PY{n}{minimum} \PY{o}{+} \PY{n}{periode}
    
\PY{n}{dates} \PY{o}{=} \PY{n}{constraintes}\PY{p}{(}\PY{l+m+mi}{50}\PY{p}{,} \PY{n}{minimum}\PY{p}{,} \PY{n}{periode}\PY{p}{,} \PY{n}{phases}\PY{p}{)}
\PY{n}{dates}\PY{o}{.}\PY{n}{head}\PY{p}{(}\PY{p}{)}
\end{Verbatim}
\end{tcolorbox}

            \begin{tcolorbox}[breakable, size=fbox, boxrule=.5pt, pad at break*=1mm, opacityfill=0]
\prompt{Out}{outcolor}{201}{\boxspacing}
\begin{Verbatim}[commandchars=\\\{\}]
                  0.000000 0.066667 0.133333 0.200000 0.266667 0.333333  \textbackslash{}
0  2020-02-23 03:35:01.864      NaN      NaN      NaN      NaN      NaN
1  2020-02-24 04:55:30.605      NaN      NaN      NaN      NaN      NaN
2                      NaN      NaN      NaN      NaN      NaN      NaN
3                      NaN      NaN      NaN      NaN      NaN      NaN
4                      NaN      NaN      NaN      NaN      NaN      NaN

  0.400000 0.466667 0.520000                 0.590000  \textbackslash{}
0      NaN      NaN      NaN                      NaN
1      NaN      NaN      NaN                      NaN
2      NaN      NaN      NaN                      NaN
3      NaN      NaN      NaN                      NaN
4      NaN      NaN      NaN  2020-02-27 23:54:01.783

                  0.670000                 0.760000                 0.830000  \textbackslash{}
0                      NaN                      NaN  2020-02-24 00:37:01.719
1                      NaN  2020-02-25 00:11:04.448  2020-02-25 01:57:30.460
2  2020-02-25 23:14:42.602  2020-02-26 01:31:33.188  2020-02-26 03:17:59.200
3  2020-02-27 00:35:11.342  2020-02-27 02:52:01.929  2020-02-27 04:38:27.941
4  2020-02-28 01:55:40.083  2020-02-28 04:12:30.669                      NaN

                  0.910000                 0.980000                 1.000000
0  2020-02-24 02:38:40.018  2020-02-24 04:25:06.030  2020-02-24 04:55:30.605
1  2020-02-25 03:59:08.759                      NaN                      NaN
2                      NaN                      NaN                      NaN
3                      NaN                      NaN                      NaN
4                      NaN                      NaN                      NaN
\end{Verbatim}
\end{tcolorbox}
        
    Il y a beaucoup de trous dans ce tableau. La cellule suivante permet de
le nettoyer, et de supprimer toutes les entrées nulles.

    \begin{tcolorbox}[breakable, size=fbox, boxrule=1pt, pad at break*=1mm,colback=cellbackground, colframe=cellborder]
\prompt{In}{incolor}{210}{\boxspacing}
\begin{Verbatim}[commandchars=\\\{\}]
\PY{n}{data} \PY{o}{=} \PY{p}{[}\PY{p}{]}
\PY{k}{for} \PY{n}{i}\PY{p}{,}\PY{n}{phase} \PY{o+ow}{in} \PY{n+nb}{enumerate}\PY{p}{(}\PY{n+nb}{list}\PY{p}{(}\PY{n}{dates}\PY{p}{)}\PY{p}{)} \PY{p}{:}
    \PY{n}{data}\PY{o}{.}\PY{n}{append}\PY{p}{(}\PY{n}{dates}\PY{p}{[}\PY{n}{phase}\PY{p}{]}\PY{o}{.}\PY{n}{dropna}\PY{p}{(}\PY{p}{)}\PY{o}{.}\PY{n}{values}\PY{o}{.}\PY{n}{tolist}\PY{p}{(}\PY{p}{)}\PY{p}{)}

\PY{n}{dates\PYZus{}sort} \PY{o}{=} \PY{n}{pd}\PY{o}{.}\PY{n}{DataFrame}\PY{p}{(}\PY{n}{data} \PY{o}{=} \PY{n}{data}\PY{p}{,} \PY{n}{index} \PY{o}{=} \PY{n}{np}\PY{o}{.}\PY{n}{around}\PY{p}{(}\PY{n}{phases}\PY{p}{,} \PY{l+m+mi}{3}\PY{p}{)}\PY{p}{)}
\PY{n}{dates\PYZus{}sort} \PY{o}{=} \PY{n}{dates\PYZus{}sort}\PY{o}{.}\PY{n}{transpose}\PY{p}{(}\PY{p}{)}
\PY{n}{dates\PYZus{}sort}
\end{Verbatim}
\end{tcolorbox}

            \begin{tcolorbox}[breakable, size=fbox, boxrule=.5pt, pad at break*=1mm, opacityfill=0]
\prompt{Out}{outcolor}{210}{\boxspacing}
\begin{Verbatim}[commandchars=\\\{\}]
                      0.000                    0.067                    0.133  \textbackslash{}
0   2020-02-23 03:35:01.864  2020-03-07 22:42:37.407  2020-03-06 23:03:30.582
1   2020-02-24 04:55:30.605  2020-03-09 00:03:06.147  2020-03-08 00:23:59.323
2   2020-03-08 22:21:44.231  2020-03-10 01:23:34.888  2020-03-09 01:44:28.063
3   2020-03-09 23:42:12.972  2020-03-11 02:44:03.628  2020-03-10 03:04:56.804
4   2020-03-11 01:02:41.712  2020-03-12 04:04:32.369  2020-03-11 04:25:25.544
5   2020-03-12 02:23:10.453  2020-03-25 21:30:45.995  2020-03-24 21:51:39.170
6   2020-03-13 03:43:39.193  2020-03-26 22:51:14.735  2020-03-25 23:12:07.911
7   2020-03-26 21:09:52.819  2020-03-28 00:11:43.476  2020-03-27 00:32:36.651
8   2020-03-27 22:30:21.560  2020-03-29 01:32:12.216  2020-03-28 01:53:05.392
9   2020-03-28 23:50:50.300  2020-03-30 02:52:40.957  2020-03-29 03:13:34.132
10  2020-03-30 01:11:19.041  2020-03-31 04:13:09.697  2020-04-11 20:39:47.759
11  2020-03-31 02:31:47.781  2020-04-12 20:18:54.583  2020-04-12 22:00:16.499
12  2020-04-01 03:52:16.522  2020-04-13 21:39:23.324  2020-04-13 23:20:45.240
13  2020-04-13 19:58:01.408  2020-04-14 22:59:52.064  2020-04-15 00:41:13.980
14  2020-04-14 21:18:30.148                     None                     None
15                     None                     None                     None
16                     None                     None                     None

                      0.200                    0.267                    0.333  \textbackslash{}
0   2020-03-05 23:24:23.758  2020-03-04 23:45:16.933  2020-03-02 22:45:41.368
1   2020-03-07 00:44:52.498  2020-03-06 01:05:45.674  2020-03-04 00:06:10.109
2   2020-03-08 02:05:21.239  2020-03-07 02:26:14.414  2020-03-05 01:26:38.849
3   2020-03-09 03:25:49.979  2020-03-08 03:46:43.155  2020-03-06 02:47:07.590
4   2020-03-10 04:46:18.720  2020-03-22 22:33:25.522  2020-03-07 04:07:36.330
5   2020-03-23 22:12:32.346  2020-03-23 23:53:54.262  2020-03-20 21:33:49.957
6   2020-03-24 23:33:01.086  2020-03-25 01:14:23.003  2020-03-21 22:54:18.697
7   2020-03-26 00:53:29.827  2020-03-26 02:34:51.743  2020-03-23 00:14:47.438
8   2020-03-27 02:13:58.567  2020-03-27 03:55:20.483  2020-03-24 01:35:16.178
9   2020-03-28 03:34:27.308  2020-04-09 21:21:34.110  2020-03-25 02:55:44.919
10  2020-04-10 21:00:40.934  2020-04-10 22:42:02.850  2020-03-26 04:16:13.659
11  2020-04-11 22:21:09.675  2020-04-12 00:02:31.591  2020-04-07 20:21:58.545
12  2020-04-12 23:41:38.415  2020-04-13 01:23:00.331  2020-04-08 21:42:27.285
13  2020-04-14 01:02:07.156  2020-04-14 02:43:29.072  2020-04-09 23:02:56.026
14  2020-04-15 02:22:35.896  2020-04-15 04:03:57.812  2020-04-11 00:23:24.766
15                     None                     None  2020-04-12 01:43:53.507
16                     None                     None  2020-04-13 03:04:22.247

                      0.400                    0.467                    0.520  \textbackslash{}
0   2020-03-01 23:06:34.544  2020-02-29 23:27:27.720  2020-02-28 23:28:04.512
1   2020-03-03 00:27:03.284  2020-03-02 00:47:56.460  2020-03-01 00:48:33.252
2   2020-03-04 01:47:32.025  2020-03-03 02:08:25.201  2020-03-02 02:09:01.993
3   2020-03-05 03:08:00.765  2020-03-04 03:28:53.941  2020-03-03 03:29:30.733
4   2020-03-06 04:28:29.506  2020-03-05 04:49:22.681  2020-03-04 04:49:59.474
5   2020-03-19 21:54:43.132  2020-03-18 22:15:36.308  2020-03-17 22:16:13.100
6   2020-03-20 23:15:11.873  2020-03-19 23:36:05.048  2020-03-18 23:36:41.841
7   2020-03-22 00:35:40.613  2020-03-21 00:56:33.789  2020-03-20 00:57:10.581
8   2020-03-23 01:56:09.354  2020-03-22 02:17:02.529  2020-03-21 02:17:39.322
9   2020-03-24 03:16:38.094  2020-03-23 03:37:31.270  2020-03-22 03:38:08.062
10  2020-04-06 20:42:51.720  2020-04-05 21:03:44.896  2020-04-04 21:04:21.688
11  2020-04-07 22:03:20.461  2020-04-06 22:24:13.636  2020-04-05 22:24:50.429
12  2020-04-08 23:23:49.201  2020-04-07 23:44:42.377  2020-04-06 23:45:19.169
13  2020-04-10 00:44:17.942  2020-04-09 01:05:11.117  2020-04-08 01:05:47.910
14  2020-04-11 02:04:46.682  2020-04-10 02:25:39.858  2020-04-09 02:26:16.650
15  2020-04-12 03:25:15.423  2020-04-11 03:46:08.598  2020-04-10 03:46:45.391
16                     None                     None                     None

                      0.590                    0.670                    0.760  \textbackslash{}
0   2020-02-27 23:54:01.783  2020-02-25 23:14:42.602  2020-02-25 00:11:04.448
1   2020-02-29 01:14:30.524  2020-02-27 00:35:11.342  2020-02-26 01:31:33.188
2   2020-03-01 02:34:59.264  2020-02-28 01:55:40.083  2020-02-27 02:52:01.929
3   2020-03-02 03:55:28.005  2020-02-29 03:16:08.823  2020-02-28 04:12:30.669
4   2020-03-16 22:42:10.371  2020-03-01 04:36:37.563  2020-03-13 22:59:13.036
5   2020-03-18 00:02:39.112  2020-03-14 22:02:51.190  2020-03-15 00:19:41.776
6   2020-03-19 01:23:07.852  2020-03-15 23:23:19.930  2020-03-16 01:40:10.517
7   2020-03-20 02:43:36.593  2020-03-17 00:43:48.671  2020-03-17 03:00:39.257
8   2020-03-21 04:04:05.333  2020-03-18 02:04:17.411  2020-03-18 04:21:07.998
9   2020-04-03 21:30:18.960  2020-03-19 03:24:46.152  2020-03-31 21:47:21.624
10  2020-04-04 22:50:47.700  2020-04-01 20:50:59.778  2020-04-01 23:07:50.365
11  2020-04-06 00:11:16.441  2020-04-02 22:11:28.518  2020-04-03 00:28:19.105
12  2020-04-07 01:31:45.181  2020-04-03 23:31:57.259  2020-04-04 01:48:47.845
13  2020-04-08 02:52:13.922  2020-04-05 00:52:25.999  2020-04-05 03:09:16.586
14  2020-04-09 04:12:42.662  2020-04-06 02:12:54.740                     None
15                     None  2020-04-07 03:33:23.480                     None
16                     None                     None                     None

                      0.830                    0.910                    0.980  \textbackslash{}
0   2020-02-24 00:37:01.719  2020-02-24 02:38:40.018  2020-02-24 04:25:06.030
1   2020-02-25 01:57:30.460  2020-02-25 03:59:08.759  2020-03-09 23:11:48.397
2   2020-02-26 03:17:59.200  2020-03-10 22:45:51.126  2020-03-11 00:32:17.137
3   2020-02-27 04:38:27.941  2020-03-12 00:06:19.866  2020-03-12 01:52:45.878
4   2020-03-11 22:04:41.567  2020-03-13 01:26:48.606  2020-03-13 03:13:14.618
5   2020-03-12 23:25:10.307  2020-03-14 02:47:17.347  2020-03-14 04:33:43.359
6   2020-03-14 00:45:39.048  2020-03-15 04:07:46.087  2020-03-27 21:59:56.985
7   2020-03-15 02:06:07.788  2020-03-28 21:33:59.714  2020-03-28 23:20:25.725
8   2020-03-16 03:26:36.529  2020-03-29 22:54:28.454  2020-03-30 00:40:54.466
9   2020-03-29 20:52:50.155  2020-03-31 00:14:57.195  2020-03-31 02:01:23.206
10  2020-03-30 22:13:18.895  2020-04-01 01:35:25.935  2020-04-01 03:21:51.947
11  2020-03-31 23:33:47.636  2020-04-02 02:55:54.676  2020-04-14 20:48:05.573
12  2020-04-02 00:54:16.376  2020-04-03 04:16:23.416  2020-04-15 22:08:34.314
13  2020-04-03 02:14:45.117  2020-04-15 20:22:08.302                     None
14  2020-04-04 03:35:13.857                     None                     None
15                     None                     None                     None
16                     None                     None                     None

                      1.000
0   2020-02-24 04:55:30.605
1   2020-03-08 22:21:44.231
2   2020-03-09 23:42:12.972
3   2020-03-11 01:02:41.712
4   2020-03-12 02:23:10.453
5   2020-03-13 03:43:39.193
6   2020-03-26 21:09:52.819
7   2020-03-27 22:30:21.560
8   2020-03-28 23:50:50.300
9   2020-03-30 01:11:19.041
10  2020-03-31 02:31:47.781
11  2020-04-01 03:52:16.522
12  2020-04-13 19:58:01.408
13  2020-04-14 21:18:30.148
14  2020-04-15 22:38:58.888
15                     None
16                     None
\end{Verbatim}
\end{tcolorbox}
        
    On exporte ensuite notre tableau au format csv, pour pouvoir ensuite le
lire facilement sur n'importe quel tableur, ou réimporter facilement le
fichier dans python

    \begin{tcolorbox}[breakable, size=fbox, boxrule=1pt, pad at break*=1mm,colback=cellbackground, colframe=cellborder]
\prompt{In}{incolor}{207}{\boxspacing}
\begin{Verbatim}[commandchars=\\\{\}]
\PY{n}{dates\PYZus{}sort}\PY{o}{.}\PY{n}{to\PYZus{}csv}\PY{p}{(}\PY{l+s+s1}{\PYZsq{}}\PY{l+s+s1}{data/dates.csv}\PY{l+s+s1}{\PYZsq{}}\PY{p}{)}
\end{Verbatim}
\end{tcolorbox}


    % Add a bibliography block to the postdoc
    
    
    
\end{document}
